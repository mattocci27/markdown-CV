% Copyright 2013 Christophe-Marie Duquesne <chmd@chmd.fr>
% Copyright 2014 Mark Szepieniec <http://github.com/mszep>
% Copyright 2017 Masatoshi Katabuchi
%
% ConText style for making a resume with pandoc. Inspired by moderncv.
%
% This CSS document is delivered to you under the CC BY-SA 3.0 License.
% https://creativecommons.org/licenses/by-sa/3.0/deed.en_US

% ######## measure #########
% # mm = 1mm = 2.85pt      #
% # cm = 10mm = 28.5pt     #
% # in = 25.4mm = 72.27pt  #
% # pt = 0.35mm = 1pt      #
% # em = width of [M]      #
% # ex = height of [x]     #
% # zw = width of [Kanji]  #
% # zh = height of [Kanji] #
% ##########################

startmode[*mkii]
  \enableregime[utf-8]
  \setupcolors[state=start]
\stopmode

\setupcolor[hex]
\definecolor[titlegrey][h=757575]
\definecolor[sectioncolor][h=0288d1]
%\definecolor[sectioncolor][h=673ab7]
\definecolor[rulecolor][h=0288d1]

% Enable hyperlinks
\setupinteraction[state=start, color=sectioncolor]

\setuppapersize [letter][letter]
\setuplayout    [width=middle, height=middle,
                 backspace=0.8in,
                 cutspace=0in,
                 topspace=0.5in,
                 bottomspace=0.8in,
                 header=0.3in,
                 footer=0in,
                 location=left]

%\setupalign[flusuleft, hz, nothyphenated,]

%\setuppagenumbering[location={header,center}]
\setupheadertexts[Masatoshi
Katabuchi, CV, Page \pagenumber{} of \totalnumberofpages]

% define font

\usesymbols [fontawesome]
%\setupsymbolset[fontawesome]

\definefontfamily[myfamily][serif][times new roman]
\definefontfamily[myfamily][sans][Optima]
%\definefontfamily[myfamily][sans][Raleway]
%\definefontfamily[myfamily][sans][Belleza]
\definefontfamily[myfamily][mono][Cousine]
\definefontfamily[myfamily][math][Cousine]

\setupbodyfont[12pt, myfamily]
%\setupbodyfont[helvetica]
\setupbodyfont[sans]
%\setupbodyfont[mono]
%\setupwhitespace[medium]
\nowhitespace
\setupinterlinespace[line=11pt]
%\setupinterlinespace[small]

\setuptolerance[horizontal,stretch] % for right margin

\setupblackrules[width=12mm, color=rulecolor]

\setuphead[chapter]      [style=\tfb]
\setuphead[section]      [style=\tfc\bf, color=titlegrey, align=middle]
%\setuphead[subsection]   [style=\tfb\bf, color=sectioncolor, align=right,
%                          before={\blank[small]\leavevmode\blackrule\hspace},
%                          after={\blank[small]}]
%\setuphead[subsection]   [style=\tfb\bf, color=sectioncolor, align=right,
%                         before={\blank[small]\leavevmode\blackrule\hspace}]
\setuphead[subsection]   [style=\tfb\bf, color=sectioncolor, align=right,
                          before={\blank[line]\leavevmode\blackrule\hspace}]
\setuphead[subsubsection][style=\bf]

%\showgrid

\setuphead[chapter, section, subsection, subsubsection][number=no]


\definedescription
  [description]
  [headstyle=normal,
   style=normal,
   afterhead{\blank[force,none]},
   location=hanging, width=3in, distance=-2.4in, margin=0in]

%\setupitemize[autointro, packed,
%              after={\blank[force,10*line]},
%              afterhead={\blank[force,10*line]}]    % prevent orphan list intro
\setupitemize[autointro, packed, 
              afterhead={\blank[force,10*line]}]    % prevent orphan list intro
%\setupitemize[autointro, packed]    % prevent orphan list intro
\setupitemize[indentnext=no]
\setupitemize[reverse]

\setupfloat[figure][default={here,nonumber}]
\setupfloat[table][default={here,nonumber}]

\setuptables[textwidth=max, HL=none]

%\setupthinrules[width=15em] % width of horizontal rules
\setupthinrules[width=15em,
	after={\blank{force,10*line}}] % width of horizontal rules

%\setupthinrules[width=15em] % width of horizontal rules

\setupdelimitedtext
  [blockquote]
  [before={\setupalign[middle]},
   indentnext=no,
  ]


\starttext
  \blank[3*medium]



\startalignment[center]
  \blank[3*medium]
  {\tfc\bf \color[titlegrey] {Masatoshi Katabuchi}}\break  Associate
Professor\break
  Xishuangbanna Tropical Botanical Garden, Chinese Academy of
Sciences\break
  Menglun, Mengla, Yunnan, 666303 China\break

\symbol[fontawesome][envelope] \goto{katabuchi@xtbg.ac.cn}[url(mailto:katabuchi@xtbg.ac.cn)]
\symbol[fontawesome][globe] \goto{mattocci27.github.io}[url(https://mattocci27.github.io)]

\stopalignment

\thinrule

\subsection[title={Research Interest},reference={research-interest}]

Community ecology, Tropical forest ecology, Functional traits, Global
change biology, Bayesian and likelihood methods

\subsection[title={Education},reference={education}]

{\bf Tohoku University, Japan}, Supervisor: Tohru Nakashizuka

\startdescription{2012}
  {\bf Ph.D.~(Life Sciences)}, Graduate School of Life Sciences
  \quotation{Community assembly and dynamics in tropical rainforests on
  the basis of functional traits}.
\stopdescription

\startdescription{2009}
  {\bf M.S. (Life Sciences)}, Graduate School of Life Sciences,
  \quotation{Reproductive ecology of a common dioecious fig and its
  pollinating wasp in a tropical secondary forest in Sarawak, Malaysia}
\stopdescription

\startdescription{2007}
  {\bf B.S. (Biology)}, Department of Biology
\stopdescription

\subsection[title={Academic & Research
Positions},reference={academic-research-positions}]

\startdescription{June 2019-}
  \useURL[url1][][][]\from[url1]

  {\bf Associate Professor}, Xishuangbanna Tropical Botanical Garden,
  Chinese Academy of Sciences, China
\stopdescription

\startdescription{August 2017-February 2019}
  \useURL[url2][][][]\from[url2]

  {\bf Postdoctoral Research Associate}, Michigan State University, USA
\stopdescription

\startdescription{January 2017-August 2017}
  \useURL[url3][][][]\from[url3]

  {\bf Visiting Scholar}, University of Florida, USA
\stopdescription

\startdescription{March 2016-January 2017}
  \useURL[url4][][][]\from[url4]

  {\bf Postdoctoral Research Associate}, University of Florida, USA
\stopdescription

\startdescription{February 2014-February 2016}
  \useURL[url5][][][]\from[url5]

  {\bf JSPS Postdoctoral Fellow for Research Abroad}, University of
  Florida, USA
\stopdescription

\startdescription{September 2012-February 2014}
  \useURL[url6][][][]\from[url6]

  {\bf Postdoctoral Research Fellow}, Xishuangbanna Tropical Botanical
  Garden, Chinese Academy of Sciences, China
\stopdescription

\startdescription{April-August 2012}
  \useURL[url7][][][]\from[url7]

  {\bf Postdoctoral Research Fellow}, Tohoku University, Japan
\stopdescription

\startdescription{April 2010-March 2012}
  \useURL[url8][][][]\from[url8]

  {\bf JSPS Research Fellow DC2}, Tohoku University, Japan
\stopdescription

\subsection[title={Funding},reference={funding}]

\startdescription{2020:}
  Chinese Academy of Sciences President's International Fellowship
  Initiative (2020FYB0003) 300,000 RMB (\$422,000)
\stopdescription

\startdescription{2014:}
  JSPS Postdoctoral Fellowship for Research Abroad (no. 25-504)
  10,512,000 JPY (\$105,000)
\stopdescription

\startdescription{2013:}
  Chinese Academy of Sciences Fellowships for Young International
  Scientists (no.2013Y1SB0002) 165,000 RMB (\$26,000)
\stopdescription

\startdescription{2010}
  Grant-in-Aid for Scientific Research for JSPS Research Fellowship (no.
  22-7035) 1,400,000 JPY (\$14,000)
\stopdescription

\subsection[title={Publications},reference={publications}]

\startenumerate[n][stopper=.]
\item
  Chen, Y.-J., Maenpuen, P., Zhang, Y.-J., Barai, K., {\bf Katabuchi,
  M.}, Gao, H., Kaewkamol, S., Tao, L.-B. & Zhang, J.-L. Quantifying
  vulnerability to embolism in tropical trees and lianas using five
  methods: Can discrepancies be explained by xylem structural traits?
  {\bf {\em New Phytologist}} 229, 805--819 (2021),
  \useURL[url9][https://doi.org/10.1111/nph.16927][][doi:10.1111/nph.16927]\from[url9].
\item
  Swenson, N. G., Hulshof, C. M., {\bf Katabuchi, M.} & Enquist, B. J.
  Long-term shifts in the functional composition and diversity of a
  tropical dry forest: A 30-yr study. {\bf {\em Ecolgical Monographs}}
  e01408 (2020),
  \useURL[url10][https://doi.org/10.1002/ecm.1408][][doi:10.1002/ecm.1408]\from[url10].
\item
  Sreekar, R., {\bf Katabuchi, M.}, Nakamura, A., Corlett, R. T., Slik,
  J. W. F., Fletcher, C., He, F., Weiblen, G. D., Shen, G., Xu, H., Sun,
  I.-F., Cao, K., Ma, K., Chang, L.-W., Cao, M., Jiang, M., Gunatilleke,
  I. A. U. N., Ong, P., Yap, S., Gunatilleke, C. V. S., Novotny, V.,
  Brockelman, W. Y., Xiang, W., Mi, X., Li, X., Wang, X., Qiao, X., Li,
  Y., Tan, S., Condit, R., Harrison, R. D. & Koh, L. P. Spatial scale
  changes the relationship between beta diversity, species richness and
  latitude. {\bf {\em Royal Society Open Science}} 5, 181168 (2018),
  \useURL[url11][https://doi.org/10.1098/rsos.181168][][doi:10.1098/rsos.181168]\from[url11].
\item
  Johnson, D. J., Needham, J., Xu, C., Massoud, E. C., Davies, S. J.,
  Anderson-Teixeira, K. J., Bunyavejchewin, S., Chambers, J. Q.,
  Chang-Yang, C.-H., Chiang, J.-M., Chuyong, G. B., Condit, R., Cordell,
  S., Fletcher, C., Giardina, C. P., Giambelluca, T. W., Gunatilleke,
  N., Gunatilleke, S., Hsieh, C.-F., Hubbell, S., Inman-Narahari, F.,
  Kassim, A. R., {\bf Katabuchi, M.}, Kenfack, D., Litton, C. M., Lum,
  S., Mohamad, M., Nasardin, M., Ong, P. S., Ostertag, R., Sack, L.,
  Swenson, N. G., Sun, I. F., Tan, S., Thomas, D. W., Thompson, J.,
  Umaña, M. N., Uriarte, M., Valencia, R., Yap, S., Zimmerman, J.,
  McDowell, N. G. & McMahon, S. M. Climate sensitive size-dependent
  survival in tropical trees. {\bf {\em Nature Ecology & Evolution}} 2,
  1436 (2018),
  \useURL[url12][https://doi.org/10.1038/s41559-018-0626-z][][doi:10.1038/s41559-018-0626-z]\from[url12].
\item
  Osnas, J. L. D., {\bf Katabuchi, M.}, Kitajima, K., Wright, S. J.,
  Reich, P. B., Van Bael, S. A., Kraft, N. J. B., Samaniego, M. J.,
  Pacala, S. W. & Lichstein, J. W. Divergent drivers of leaf trait
  variation within species, among species, and among functional groups.
  {\bf {\em Proceedings of the National Academy of Sciences}} 115,
  5480--5485 (2018),
  \useURL[url13][https://doi.org/10.1073/pnas.1803989115][][doi:10.1073/pnas.1803989115]\from[url13].
\item
  Li, Y., Shipley, B., Price, J. N., Dantas, V. de L., Tamme, R.,
  Westoby, M., Siefert, A., Schamp, B. S., Spasojevic, M. J., Jung, V.,
  Laughlin, D. C., Richardson, S. J., Bagousse-Pinguet, Y. L., Schöb,
  C., Gazol, A., Prentice, H. C., Gross, N., Overton, J., Cianciaruso,
  M. V., Louault, F., Kamiyama, C., Nakashizuka, T., Hikosaka, K.,
  Sasaki, T., Katabuchi, M., Frenette Dussault, C., Gaucherand, S.,
  Chen, N., Vandewalle, M. & Batalha, M. A. Habitat Filtering Determines
  the Functional Niche Occupancy of Plant Communities Worldwide.
  {\bf {\em Journl of Ecology}} 3, 808--809 (2017),
  \useURL[url14][https://doi.org/10.1111/1365-2745.12802][][doi:10.1111/1365-2745.12802]\from[url14].
\item
  {\bf Katabuchi, M.}, Wright, S. J., Swenson, N. G., Feeley, K. J.,
  Condit, R., Hubbell, S. P. & Davies, S. J. Contrasting outcomes of
  species- and community-level analyses of the temporal consistency of
  functional composition. {\bf {\em Ecology}} 98, 2273--2280 (2017),
  \useURL[url15][https://doi.org/10.1002/ecy.1952][][doi:10.1002/ecy.1952]\from[url15].
\item
  Siefert, A., Violle, C., Chalmandrier, L., Albert, C. H., Taudiere,
  A., Fajardo, A., Aarssen, L. W., Baraloto, C., Carlucci, M. B.,
  Cianciaruso, M. V., L. Dantas, V. de, Bello, F. de, Duarte, L. D. S.,
  Fonseca, C. R., Freschet, G. T., Gaucherand, S., Gross, N., Hikosaka,
  K., Jackson, B., Jung, V., Kamiyama, C., {\bf Katabuchi, M.}, Kembel,
  S. W., Kichenin, E., Kraft, N. J. B., Lagerström, A.,
  Bagousse-Pinguet, Y. L., Li, Y., Mason, N., Messier, J., Nakashizuka,
  T., Overton, J. McC., Peltzer, D. A., Pérez-Ramos, I. M., Pillar, V.
  D., Prentice, H. C., Richardson, S., Sasaki, T., Schamp, B. S., Schöb,
  C., Shipley, B., Sundqvist, M., Sykes, M. T., Vandewalle, M. & Wardle,
  D. A. A global meta-analysis of the relative extent of intraspecific
  trait variation in plant communities. {\bf {\em Ecology Letters}} 18,
  1406--1419 (2015),
  \useURL[url16][https://doi.org/10.1111/ele.12508][][doi:10.1111/ele.12508]\from[url16].
\item
  {\bf Katabuchi, M.} LeafArea: an R package for rapid digital image
  analysis of leaf area. {\bf {\em Ecological Research}} 30, 1073--1077
  (2015),
  \useURL[url17][https://doi.org/10.1007/s11284-015-1307-x][][doi:10.1007/s11284-015-1307-x]\from[url17].
\item
  Ashton, L. A., Nakamura, A., Basset, Y., Burwell, C. J., Cao, M.,
  Eastwood, R., Odell, E., Oliveira, E. G. de, Hurley, K.,
  {\bf Katabuchi, M.}, Maunsell, S., McBroom, J., Schmidl, J., Sun, Z.,
  Tang, Y., Whitaker, T., Laidlaw, M. J., McDonald, W. J. F. & Kitching,
  R. L. Vertical stratification of moths across elevation and latitude.
  {\bf {\em Journal of Biogeography}} 43, 59--69 (2016),
  \useURL[url18][https://doi.org/10.1111/jbi.12616][][doi:10.1111/jbi.12616]\from[url18].
\item
  Nakamura, A., Burwell, C. J., Ashton, L. A., Laidlaw, M. J.,
  Katabuchi, M. & Kitching, R. L. Identifying indicator species of
  elevation: Comparing the utility of woody plants, ants and moths for
  long-term monitoring. {\bf {\em Austral Ecology}} 41, 179--188 (2016),
  \useURL[url19][https://doi.org/10.1111/aec.12291][][doi:10.1111/aec.12291]\from[url19].
\item
  Nakamura, A., Burwell, C. J., Lambkin, C. L., {\bf Katabuchi, M.},
  Mcdougall, A., Raven, R. J. & Neldner, V. J. The Role of Human
  Disturbance in Island Biogeography of Arthropods and Plants: An
  Information Theoretic Approach. {\bf {\em Journal of Biogeography}}
  42, 1406--1417 (2015),
  \useURL[url20][https://doi.org/10.1111/jbi.12520][][doi:10.1111/jbi.12520]\from[url20].
\item
  Deng, X., Mohandass, D., {\bf Katabuchi, M.}, Hughes, A. C. & Roubik,
  D. W. Impact of Striped-Squirrel Nectar-Robbing Behaviour on Gender
  Fitness in Alpinia roxburghii Sweet (Zingiberaceae). {\bf {\em PLoS
  ONE}} 10, (2015),
  \useURL[url21][https://doi.org/10.1371/journal.pone.0144585][][doi:10.1371/journal.pone.0144585]\from[url21].
\item
  Hikosaka, K., Sasaki, T., Kamiyama, C., {\bf Katabuchi, M.}, Oikawa,
  S., Shimazaki, M., Kimura, K. & Nakshizuka, T. Understanding of
  species niche, coexistence and extinction based on functional traits :
  Approaches from community and physiological ecology for subalpine
  moorland plant communities {[}in Japanese{]}.
  {\bf {\em Chikyu-Kankyo}} 19, 33--46 (2014).
\item
  Kamiyama, C., {\bf Katabuchi, M.}, Sasaki, T., Shimazaki, M.,
  Nakashizuka, T. & Hikosaka, K. Leaf-trait responses to environmental
  gradients in moorland communities: Contribution of intraspecific
  variation, species replacement and functional group replacement.
  {\bf {\em Ecological Research}} 29, 607--617 (2014),
  \useURL[url22][https://doi.org/10.1007/s11284-014-1148-z][][doi:10.1007/s11284-014-1148-z]\from[url22].
\item
  Sasaki, T., {\bf Katabuchi, M.}, Kamiyama, C., Shimazaki, M.,
  Nakashizuka, T. & Hikosaka, K. Vulnerability of moorland plant
  communities to environmental change: Consequences of realistic species
  loss on functional diversity. {\bf {\em Journal of Applied Ecology}}
  51, 299--308 (2014),
  \useURL[url23][https://doi.org/10.1111/1365-2664.12192][][doi:10.1111/1365-2664.12192]\from[url23].
\item
  Cádiz, A., Nagata, N., {\bf Katabuchi, M.}, Dı́az, L. M.,
  Echenique-Dı́az, L. M., Akashi, H. D., Makino, T. & Kawata, M. Relative
  importance of habitat use, range expansion, and speciation in local
  species diversity of {\em Anolis} lizards in Cuba.
  {\bf {\em Ecosphere}} 4, art78 (2013),
  \useURL[url24][https://doi.org/10.1890/es12-00383.1][][doi:10.1890/es12-00383.1]\from[url24].
\item
  Asanok, L., Marod, D., Duengkae, P., Pranmongkol, U., Kurokawa, H.,
  Aiba, M., {\bf Katabuchi, M.} & Nakashizuka, T. Relationships between
  functional traits and the ability of forest tree species to
  reestablish in secondary forest and enrichment plantations in the
  uplands of northern Thailand. {\bf {\em Forest Ecology and
  Management}} 296, 9--23 (2013),
  \useURL[url25][https://doi.org/10.1016/j.foreco.2013.01.029][][doi:10.1016/j.foreco.2013.01.029]\from[url25].
\item
  Sasaki, T., {\bf Katabuchi, M.}, Kamiyama, C., Shimazaki, M.,
  Nakashizuka, T. & Hikosaka, K. Variations in Species Composition of
  Moorland Plant Communities Along Environmental Gradients Within a
  Subalpine Zone in Northern Japan. {\bf {\em Wetlands}} 33, 269--277
  (2013),
  \useURL[url26][https://doi.org/10.1007/s13157-013-0380-6][][doi:10.1007/s13157-013-0380-6]\from[url26].
\item
  Aiba, M., {\bf Katabuchi, M.}, Takafumi, H., Matsuzaki, S. S., Sasaki,
  T. & Hiura, T. Robustness of trait distribution metrics for community
  assembly studies under the uncertainties of assembly processes.
  {\bf {\em Ecology}} 94, 2873--2885 (2013),
  \useURL[url27][https://doi.org/10.1890/13-0269.1][][doi:10.1890/13-0269.1]\from[url27].
\item
  Sasaki, T., {\bf Katabuchi, M.}, Kamiyama, C., Shimazaki, M.,
  Nakashizuka, T. & Hikosaka, K. Diversity partitioning of moorland
  plant communities across hierarchical spatial scales.
  {\bf {\em Biodiversity and Conservation}} 21, 1577--1588 (2012),
  \useURL[url28][https://doi.org/10.1007/s10531-012-0265-7][][doi:10.1007/s10531-012-0265-7]\from[url28].
\item
  Sasaki, T., {\bf Katabuchi, M.}, Kamiyama, C., Shimazaki, M.,
  Nakashizuka, T. & Hikosaka, K. Nestedness and niche-based species loss
  in moorland plant communities. {\bf {\em Oikos}} 121, 1783--1790
  (2012),
  \useURL[url29][https://doi.org/10.1111/j.1600-0706.2012.20152.x][][doi:10.1111/j.1600-0706.2012.20152.x]\from[url29].
\item
  {\bf Katabuchi, M.}, Kurokawa, H., Davies, S. J., Tan, S. &
  Nakashizuka, T. Soil resource availability shapes community trait
  structure in a species-rich dipterocarp forest. {\bf {\em Journal of
  Ecology}} 100, 643--651 (2012),
  \useURL[url30][https://doi.org/10.1111/j.1365-2745.2011.01937.x][][doi:10.1111/j.1365-2745.2011.01937.x]\from[url30].
\item
  {\bf Katabuchi, M.}, Isagi, Y. & Nakashizuka, T. Development of 17
  microsatellite markers for {\em Ceratosolen Constrictus}, the
  pollinating fig wasp of {\em Ficus Fistulosa}. {\bf {\em Molecular
  Ecology Resources}} 8, 840--842 (2008),
  \useURL[url31][https://doi.org/10.1111/j.1755-0998.2007.02084.x][][doi:10.1111/j.1755-0998.2007.02084.x]\from[url31].
\item
  {\bf Katabuchi, M.}, Harrison, R. D. & Nakashizuka, T. Documenting the
  effect of foundress number in a dioecious fig, {\em Ficus Fistulosa},
  in Malaysia. {\bf {\em Biotropica}} 40, 457--461 (2008),
  \useURL[url32][https://doi.org/10.1111/j.1744-7429.2008.00405.x][][doi:10.1111/j.1744-7429.2008.00405.x]\from[url32].
\stopenumerate

\subsection[title={Manuscript in
review/revision/preprint},reference={manuscript-in-reviewrevisionpreprint}]

\startenumerate[n,packed][stopper=.]
\item
  {\bf Katabuchi M,} Kitajima K, S.J. Wright, S.A. Van Bael, J.L.D.
  Osnas and J.W. Lichstein. Decomposing leaf mass into photosynthetic
  and structural components explains divergent patterns of trait
  variation within and among plant species. {\em bioRxiv} 116855; doi:
  https://doi.org/10.1101/116855
\stopenumerate

\subsection[title={Book},reference={book}]

\startenumerate[n,packed][stopper=.]
\item
  Hikosaka K., Sasaki T., Kamiyama C., {\bf Katabuchi M.}, Oikawa S.,
  Shimazaki M., Kimura H. and Nakashizuka T. (2016), Trait-Based
  Approaches for Understanding Species Niche, Coexistence, and
  Functional Diversity in Subalpine Moorlands. Structure and Function of
  Mountain Ecosystems in Japan.
\stopenumerate

\subsection[title={Software},reference={software}]

\startdescription{2020}
  {\bf ztpln}: Zero-Truncated Poisson Lognormal Distribution, R CRAN
  package
\stopdescription

\startdescription{2015}
  {\bf LeafArea}: Rapid Digital Image Analysis of Leaf Area, R CRAN
  package
\stopdescription

\startdescription{2015}
  {\bf mglmn}: Model Averaging fro Multivariate GLM with Null Models, R
  CRAN package
\stopdescription

\subsection[title={Teaching Experience},reference={teaching-experience}]

\startdescription{2019-}
  How do we measure and study biological diveristy?, Advanced Field
  course in Ecology and Conservation, Xishuangbanna Tropical Botanical
  Garden, Chinese Academy of Sciences, China
\stopdescription

\startdescription{2013}
  Experimental Design and Data Analysis, ATBC Asia-Pacific chapter,
  Banda Aceh, Indonesia
\stopdescription

\startdescription{2013}
  R and generalized linear models, Xishuangbanna Tropical Botanical
  Garden, Chinese Academy of Sciences, China
\stopdescription

\subsection[title={Service},reference={service}]

\startdescription{2015-}
  Working Group Member of INNGE, International Network of
  Next-Generation Ecologists
\stopdescription

\startdescription{2015-}
  INNGE manager, The Japan Society of Tropical Ecology
\stopdescription

\startdescription{2013-}
  Statistical editor for Lepcey, The Journal of Tropical Asian
  Entomology
\stopdescription

Reviewer for journals (\lettertilde{} 12 per year):
\startcolumns[n=2]
- AoB Plants

- Biotropica

- BMC Evolutionary Biology

- Ecological Research

- Ecology Letters

- Ecosphere

- Forest Ecology and Management

- Frontiers in Plant Science

- Functional Ecology

- Journal of Applied Ecology

- Journal of Ecology

- Journal of Plant Research

- Journal of Sustainable Forestry

- Journal of Vegetation Science

- Nature Communications

- Oecologia

- Oikos

- Plant Ecology

- PLOS ONE

- Scientific Reports
\stopcolumns

\subsection[title={Awards and Prizes},reference={awards-and-prizes}]

\startdescription{2012}
  Dean's award for excellence, Graduate School of Life Sciences
  (Doctoral Program), Tohoku University
\stopdescription

\startdescription{2009}
  Dean's award for excellence, Graduate School of Life Sciences
  (Master's Program), Tohoku University
\stopdescription

\subsection[title={Invited Seminars},reference={invited-seminars}]

\startdescription{2021}
  Introduction to visualization using ggplot2, International Institute
  of Tropical Agriculture, Kenya
\stopdescription

\startdescription{2018}
  A quantitive model for divergent drivers of leaf trait variation
  within and among plant species, Xishuangbanna Tropical Botanical
  Garden, China
\stopdescription

\startdescription{2018}
  Species coexistence and functional traits. Graduate School of Life
  Sciences, Tohoku University, Japan
\stopdescription

\startdescription{2018}
  Species coexistence and functional traits. Graduate School of
  Environment and Information Sciences, Yokoyama National University,
  Japan
\stopdescription

\startdescription{2016}
  Decomposing leaf mass into photosynthetic and structural components
  explains divergent patterns of trait variation within and among plant
  species. Department of Ecology and Evolutionary Biology, Princeton
  University, NJ
\stopdescription

\startdescription{2013}
  Additive partitioning of functional diversity in plant communities,
  Xishuangbanna Tropical Botanical Garden, China
\stopdescription

\subsection[title={International
Meetings},reference={international-meetings}]

{\bf Invited}

\startdescription{2015}
  {\bf Masatoshi Katabuchi}, \quotation{Future perspective for
  functional trait research}, Organized session on
  \quotation{Achievements and outstanding questions for trait-based
  forest ecology} , 62nd ESJ Annual meeting, Kagoshima, Japan
\stopdescription

\startdescription{2014}
  {\bf Masatoshi Katabuchi (Organizer)}, \quotation{Seedling dynamics
  and its consequence to community assembly processes in a tropical
  forest}, Symposium on \quotation{Integrating the structure and
  dynamics of tree communities to uncover the mechanisms underlying
  species coexistence}, 61st ESJ Annual meeting, Hiroshima, Japan
\stopdescription

\startdescription{2011}
  {\bf Masatoshi Katabuchi} and Hiroko Kurokawa \quotation{Community
  assembly and demography of tropical tree species}, Workshop on
  \quotation{Life history regulation of forest trees: towards
  cross-biome analysis}, Sapporo, Japan
\stopdescription

{\bf Submitted}

\startdescription{2015}
  {\bf Masatoshi Katabuchi}, Kaoru Kitajima, Joesph Wright, Jeanne Osnas
  and Jeremy Lichstein \quotation{Decomposing leaf mass into
  photosynthetic and structural components explains diverge patterns of
  trait variation within vs.~among plant communities}, 100th ESA Annual
  meeting, Baltimore, USA
\stopdescription

\startdescription{2014}
  {\bf Masatoshi Katabuchi}, Takehiro Sasaki, Chiho Kamiyama, Masaya
  Shimazaki, Tohru Nakashizuka and Kouki Hikosaka
  \quotation{Intraspecific trait variations affect spatial patterns of
  functional diversity among moorland plant communities}, 99th ESA
  Annual meeting, Sacramento, USA.
\stopdescription

\startdescription{2013}
  {\bf Masatoshi Katabuchi}, Sylvester Tan and Tohru Nakashizuka
  \quotation{Negative density dependence reduces survival rates of
  common tree species in a species-rich tropical rain forest}, ATBC, San
  Jose, Costa Rica
\stopdescription

\startdescription{2012}
  {\bf Masatoshi Katabuchi}, Hiroko Kurokawa, Stuart Davies, Sylvester
  Tan and Tohru Nakashizuka \quotation{Soil resource availability shapes
  community trait structure in a species-rich dipterocarp forest}, ATBC
  Asia-Pacific chapter, XTBG, China
\stopdescription

\startdescription{2011}
  {\bf Masatoshi Katabuchi}, Hiroko Kurokawa, Sylvester Tan, Stuart J.
  Davies, and Tohru Nakashizuka, \quotation{Nonrandom abundance
  distributions of a tropical tree community: evidence from traits
  correlation at a local scale}. ATBC, Arusha, Tanzania
\stopdescription

\startdescription{2010}
  {\bf Masatoshi Katabuchi}, Hiroko Kurokawa, Sylvester Tan, and Tohru
  Nakashizuka, \quotation{Soil variation and functional trait assembly
  in a diverse dipterocarp forest}. 95th ESA Annual meeting, Pittsburgh,
  USA
\stopdescription

\startdescription{2009}
  {\bf Masatoshi Katabuchi}, Jun Yokoyama, and Tohru Nakashizuka,
  \quotation{Inbreeding and mating structure in a dioecious-fig
  pollinating wasp}. ATBC, Marburg, Germany
\stopdescription

\subsection[title={Outreach},reference={outreach}]

\startdescription{2015}
  Masatoshi Katabuchi \quotation{Leaves and tropical rainforests}.
  Invited by the Japanese Association of Interdisciplinary Study Group
  in University Florida
\stopdescription

\subsection[title={Languages},reference={languages}]

Japanese (native), English (fluent), Malay (basic), Chinese-Mandarin
(basic)

\subsection[title={References for Masatoshi
Katabuchi},reference={references-for-masatoshi-katabuchi}]

Tohru Nakashizuka

Research Institute for Humanity and Nature

457-4, Motoyama, Kamigamo, Kita-ku, Kyoto, 603-8047 Japan

\useURL[url9][mailto:toron@chikyu.ac.jp][][toron@chikyu.ac.jp]\from[url9]

+81(0)75-707-2341

$~$

Kaoru Kitajima

Kyoto University

Graduate School of Agriculture,

Kitashirakawa Oiwake-cho, Kyoto, 606-8520 Japan

\useURL[url10][mailto:kaoruk@kais.kyoto-u.ac.jp][][kaoruk@kais.kyoto-u.ac.jp]\from[url10]

+81(0)75-753-6360

$~$

Jeremy Lichstein

University of Florida

Department of Biology

Gainesville, FL 32611 USA

\useURL[url11][mailto:jlichstein@ufl.edu][][jlichstein@ufl.edu]\from[url11]

+1 (352) 392-1540

$~$

Christopher Klausmeier

W. K. Kellogg Biological Station

Michigan State University

Hickory Corners, MI 49060 USA

\useURL[url12][mailto:klausme1@msu.edu][][klausme1@msu.edu]\from[url12]
