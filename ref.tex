% Options for packages loaded elsewhere
\PassOptionsToPackage{unicode}{hyperref}
\PassOptionsToPackage{hyphens}{url}
%
\documentclass[
]{article}
\usepackage{lmodern}
\usepackage{amssymb,amsmath}
\usepackage{ifxetex,ifluatex}
\ifnum 0\ifxetex 1\fi\ifluatex 1\fi=0 % if pdftex
  \usepackage[T1]{fontenc}
  \usepackage[utf8]{inputenc}
  \usepackage{textcomp} % provide euro and other symbols
\else % if luatex or xetex
  \usepackage{unicode-math}
  \defaultfontfeatures{Scale=MatchLowercase}
  \defaultfontfeatures[\rmfamily]{Ligatures=TeX,Scale=1}
\fi
% Use upquote if available, for straight quotes in verbatim environments
\IfFileExists{upquote.sty}{\usepackage{upquote}}{}
\IfFileExists{microtype.sty}{% use microtype if available
  \usepackage[]{microtype}
  \UseMicrotypeSet[protrusion]{basicmath} % disable protrusion for tt fonts
}{}
\makeatletter
\@ifundefined{KOMAClassName}{% if non-KOMA class
  \IfFileExists{parskip.sty}{%
    \usepackage{parskip}
  }{% else
    \setlength{\parindent}{0pt}
    \setlength{\parskip}{6pt plus 2pt minus 1pt}}
}{% if KOMA class
  \KOMAoptions{parskip=half}}
\makeatother
\usepackage{xcolor}
\IfFileExists{xurl.sty}{\usepackage{xurl}}{} % add URL line breaks if available
\IfFileExists{bookmark.sty}{\usepackage{bookmark}}{\usepackage{hyperref}}
\hypersetup{
  hidelinks,
  pdfcreator={LaTeX via pandoc}}
\urlstyle{same} % disable monospaced font for URLs
\setlength{\emergencystretch}{3em} % prevent overfull lines
\providecommand{\tightlist}{%
  \setlength{\itemsep}{0pt}\setlength{\parskip}{0pt}}
\setcounter{secnumdepth}{-\maxdimen} % remove section numbering
\makeatletter
\@ifpackageloaded{subfig}{}{\usepackage{subfig}}
\@ifpackageloaded{caption}{}{\usepackage{caption}}
\captionsetup[subfloat]{margin=0.5em}
\AtBeginDocument{%
\renewcommand*\figurename{Figure}
\renewcommand*\tablename{Table}
}
\AtBeginDocument{%
\renewcommand*\listfigurename{List of Figures}
\renewcommand*\listtablename{List of Tables}
}
\@ifpackageloaded{float}{}{\usepackage{float}}
\floatstyle{ruled}
\@ifundefined{c@chapter}{\newfloat{codelisting}{h}{lop}}{\newfloat{codelisting}{h}{lop}[chapter]}
\floatname{codelisting}{Listing}
\newcommand*\listoflistings{\listof{codelisting}{List of Listings}}
\makeatother
\newlength{\cslhangindent}
\setlength{\cslhangindent}{1.5em}
\newenvironment{cslreferences}%
  {}%
  {\par}

\author{}
\date{}

\begin{document}

1 2 3 4 5 6 7 8 9 10 11 12 13 14 15 16 17 18 19 20 21 22 23 24

\hypertarget{ue-seperate-md-files}{%
\section{Ue seperate md files}\label{ue-seperate-md-files}}

\begin{verbatim}
pandoc ref.md -F pandoc-crossref -F pandoc-citeproc \
--bibliography=ref.bib \
--csl=my_ref.csl \
--template=eisvogel2.tex \
--default-image-extension=pdf \
--pdf-engine=xelatex \
-s -o ref.pdf

pandoc ref.md -F pandoc-crossref -F pandoc-citeproc \
--bibliography=ref.bib \
--csl=my_ref.csl \
--default-image-extension=pdf \
--template=eisvogel2.tex \
--pdf-engine=xelatex \
-s -o ref.tex

pandoc ref.tex -o ref_citeproc.md

#Rscript ref.r 
rm -f ref_citeproc_edit.md
python ref.py 

pandoc -s CV1.md ref_citeproc_edit.md CV2.md -F pandoc-citeproc \
--template CV_template.tex \
--from markdown --to context \
--pdf-engine=xelatex \
--bibliography=ref.bib \
--csl=my_ref.csl \
-V papersize=letter \
-o Katabuchi_CV.tex; \
context Katabuchi_CV.tex
\end{verbatim}

\hypertarget{use-a-single-md-file}{%
\section{Use a single md file}\label{use-a-single-md-file}}

\begin{itemize}
\tightlist
\item
  not working
\end{itemize}

\begin{verbatim}
pandoc CV.md --standalone -F pandoc-citeproc \
--template CV_template.tex \
--from markdown --to context \
--bibliography=ref.bib \
--csl=my_ref.csl \
--pdf-engine=xelatex \
-V papersize=letter \
-o Katabuchi_CV.tex
\end{verbatim}

\hypertarget{refs}{%
\section*{Refs}\label{refs}}
\addcontentsline{toc}{section}{Refs}

\hypertarget{refs}{}
\begin{cslreferences}
\leavevmode\hypertarget{ref-Swenson2020}{}%
1. Swenson, N. G., Hulshof, C. M., Katabuchi, M. \& Enquist, B. J.
Long-term shifts in the functional composition and diversity of a
tropical dry forest: A 30-yr study. \emph{\textbf{Ecolgical Monographs}}
e01408 (2020),
doi:\href{https://doi.org/10.1002/ecm.1408}{10.1002/ecm.1408}.

\leavevmode\hypertarget{ref-Sreekar2018}{}%
2. Sreekar, R., Katabuchi, M., Nakamura, A., Corlett, R. T., Slik, J. W.
F., Fletcher, C., He, F., Weiblen, G. D., Shen, G., Xu, H., Sun, I.-F.,
Cao, K., Ma, K., Chang, L.-W., Cao, M., Jiang, M., Gunatilleke, I. A. U.
N., Ong, P., Yap, S., Gunatilleke, C. V. S., Novotny, V., Brockelman, W.
Y., Xiang, W., Mi, X., Li, X., Wang, X., Qiao, X., Li, Y., Tan, S.,
Condit, R., Harrison, R. D. \& Koh, L. P. Spatial scale changes the
relationship between beta diversity, species richness and latitude.
\emph{\textbf{Royal Society Open Science}} 5, 181168 (2018),
doi:\href{https://doi.org/10.1098/rsos.181168}{10.1098/rsos.181168}.

\leavevmode\hypertarget{ref-Johnson2018}{}%
3. Johnson, D. J., Needham, J., Xu, C., Massoud, E. C., Davies, S. J.,
Anderson-Teixeira, K. J., Bunyavejchewin, S., Chambers, J. Q.,
Chang-Yang, C.-H., Chiang, J.-M., Chuyong, G. B., Condit, R., Cordell,
S., Fletcher, C., Giardina, C. P., Giambelluca, T. W., Gunatilleke, N.,
Gunatilleke, S., Hsieh, C.-F., Hubbell, S., Inman-Narahari, F., Kassim,
A. R., Katabuchi, M., Kenfack, D., Litton, C. M., Lum, S., Mohamad, M.,
Nasardin, M., Ong, P. S., Ostertag, R., Sack, L., Swenson, N. G., Sun,
I. F., Tan, S., Thomas, D. W., Thompson, J., Umaña, M. N., Uriarte, M.,
Valencia, R., Yap, S., Zimmerman, J., McDowell, N. G. \& McMahon, S. M.
Climate sensitive size-dependent survival in tropical trees.
\emph{\textbf{Nature Ecology \& Evolution}} 2, 1436 (2018),
doi:\href{https://doi.org/10.1038/s41559-018-0626-z}{10.1038/s41559-018-0626-z}.

\leavevmode\hypertarget{ref-Osnas2018}{}%
4. Osnas, J. L. D., Katabuchi, M., Kitajima, K., Wright, S. J., Reich,
P. B., Van Bael, S. A., Kraft, N. J. B., Samaniego, M. J., Pacala, S. W.
\& Lichstein, J. W. Divergent drivers of leaf trait variation within
species, among species, and among functional groups.
\emph{\textbf{Proceedings of the National Academy of Sciences}} 115,
5480--5485 (2018),
doi:\href{https://doi.org/10.1073/pnas.1803989115}{10.1073/pnas.1803989115}.

\leavevmode\hypertarget{ref-Li2017}{}%
5. Li, Y., Shipley, B., Price, J. N., Dantas, V. de L., Tamme, R.,
Westoby, M., Siefert, A., Schamp, B. S., Spasojevic, M. J., Jung, V.,
Laughlin, D. C., Richardson, S. J., Bagousse-Pinguet, Y. L., Schöb, C.,
Gazol, A., Prentice, H. C., Gross, N., Overton, J., Cianciaruso, M. V.,
Louault, F., Kamiyama, C., Nakashizuka, T., Hikosaka, K., Sasaki, T.,
Katabuchi, M., Frenette Dussault, C., Gaucherand, S., Chen, N.,
Vandewalle, M. \& Batalha, M. A. Habitat Filtering Determines the
Functional Niche Occupancy of Plant Communities Worldwide.
\emph{\textbf{Journl of Ecology}} 3, 808--809 (2017),
doi:\href{https://doi.org/10.1111/1365-2745.12802}{10.1111/1365-2745.12802}.

\leavevmode\hypertarget{ref-Katabuchi2017}{}%
6. Katabuchi, M., Wright, S. J., Swenson, N. G., Feeley, K. J., Condit,
R., Hubbell, S. P. \& Davies, S. J. Contrasting outcomes of species- and
community-level analyses of the temporal consistency of functional
composition. \emph{\textbf{Ecology}} 98, 2273--2280 (2017),
doi:\href{https://doi.org/10.1002/ecy.1952}{10.1002/ecy.1952}.

\leavevmode\hypertarget{ref-Siefert2015}{}%
7. Siefert, A., Violle, C., Chalmandrier, L., Albert, C. H., Taudiere,
A., Fajardo, A., Aarssen, L. W., Baraloto, C., Carlucci, M. B.,
Cianciaruso, M. V., L. Dantas, V. de, Bello, F. de, Duarte, L. D. S.,
Fonseca, C. R., Freschet, G. T., Gaucherand, S., Gross, N., Hikosaka,
K., Jackson, B., Jung, V., Kamiyama, C., Katabuchi, M., Kembel, S. W.,
Kichenin, E., Kraft, N. J. B., Lagerström, A., Bagousse-Pinguet, Y. L.,
Li, Y., Mason, N., Messier, J., Nakashizuka, T., Overton, J. M.,
Peltzer, D. A., Pérez-Ramos, I. M., Pillar, V. D., Prentice, H. C.,
Richardson, S., Sasaki, T., Schamp, B. S., Schöb, C., Shipley, B.,
Sundqvist, M., Sykes, M. T., Vandewalle, M. \& Wardle, D. A. A global
meta-analysis of the relative extent of intraspecific trait variation in
plant communities. \emph{\textbf{Ecology Letters}} 18, 1406--1419
(2015), doi:\href{https://doi.org/10.1111/ele.12508}{10.1111/ele.12508}.

\leavevmode\hypertarget{ref-Katabuchi2015}{}%
8. Katabuchi, M. LeafArea: an R package for rapid digital image analysis
of leaf area. \emph{\textbf{Ecological Research}} 30, 1073--1077 (2015),
doi:\href{https://doi.org/10.1007/s11284-015-1307-x}{10.1007/s11284-015-1307-x}.

\leavevmode\hypertarget{ref-Ashton2016}{}%
9. Ashton, L. A., Nakamura, A., Basset, Y., Burwell, C. J., Cao, M.,
Eastwood, R., Odell, E., Oliveira, E. G. de, Hurley, K., Katabuchi, M.,
Maunsell, S., McBroom, J., Schmidl, J., Sun, Z., Tang, Y., Whitaker, T.,
Laidlaw, M. J., McDonald, W. J. F. \& Kitching, R. L. Vertical
stratification of moths across elevation and latitude.
\emph{\textbf{Journal of Biogeography}} 43, 59--69 (2016),
doi:\href{https://doi.org/10.1111/jbi.12616}{10.1111/jbi.12616}.

\leavevmode\hypertarget{ref-Nakamura2016}{}%
10. Nakamura, A., Burwell, C. J., Ashton, L. A., Laidlaw, M. J.,
Katabuchi, M. \& Kitching, R. L. Identifying indicator species of
elevation: Comparing the utility of woody plants, ants and moths for
long-term monitoring. \emph{\textbf{Austral Ecology}} 41, 179--188
(2016), doi:\href{https://doi.org/10.1111/aec.12291}{10.1111/aec.12291}.

\leavevmode\hypertarget{ref-Nakamura2015}{}%
11. Nakamura, A., Burwell, C. J., Lambkin, C. L., Katabuchi, M.,
Mcdougall, A., Raven, R. J. \& Neldner, V. J. The Role of Human
Disturbance in Island Biogeography of Arthropods and Plants: An
Information Theoretic Approach. \emph{\textbf{J. Biogeogr.}} 42,
1406--1417 (2015),
doi:\href{https://doi.org/10.1111/jbi.12520}{10.1111/jbi.12520}.

\leavevmode\hypertarget{ref-Deng2015}{}%
12. Deng, X., Mohandass, D., Katabuchi, M., Hughes, A. C. \& Roubik, D.
W. Impact of Striped-Squirrel Nectar-Robbing Behaviour on Gender Fitness
in Alpinia roxburghii Sweet (Zingiberaceae). \emph{\textbf{PLoS ONE}}
10, (2015),
doi:\href{https://doi.org/10.1371/journal.pone.0144585}{10.1371/journal.pone.0144585}.

\leavevmode\hypertarget{ref-Hikosaka2014}{}%
13. Hikosaka, K., Sasaki, T., Kamiyama, C., Katabuchi, M., Oikawa, S.,
Shimazaki, M., Kimura, K. \& Nakshizuka, T. Understanding of species
niche, coexistence and extinction based on functional traits :
Approaches from community and physiological ecology for subalpine
moorland plant communities {[}in Japanese{]}.
\emph{\textbf{Chikyu-Kankyo}} 19, 33--46 (2014).

\leavevmode\hypertarget{ref-Kamiyama2014}{}%
14. Kamiyama, C., Katabuchi, M., Sasaki, T., Shimazaki, M., Nakashizuka,
T. \& Hikosaka, K. Leaf-trait responses to environmental gradients in
moorland communities: Contribution of intraspecific variation, species
replacement and functional group replacement. \emph{\textbf{Ecological
Research}} 29, 607--617 (2014),
doi:\href{https://doi.org/10.1007/s11284-014-1148-z}{10.1007/s11284-014-1148-z}.

\leavevmode\hypertarget{ref-Sasaki2014}{}%
15. Sasaki, T., Katabuchi, M., Kamiyama, C., Shimazaki, M., Nakashizuka,
T. \& Hikosaka, K. Vulnerability of moorland plant communities to
environmental change: Consequences of realistic species loss on
functional diversity. \emph{\textbf{Journal of Applied Ecology}} 51,
299--308 (2014),
doi:\href{https://doi.org/10.1111/1365-2664.12192}{10.1111/1365-2664.12192}.

\leavevmode\hypertarget{ref-Cdiz2013}{}%
16. Cádiz, A., Nagata, N., Katabuchi, M., Dı́az, L. M., Echenique-Dı́az,
L. M., Akashi, H. D., Makino, T. \& Kawata, M. Relative importance of
habitat use, range expansion, and speciation in local species diversity
of Anolis lizards in Cuba. \emph{\textbf{Ecosphere}} 4, art78 (2013),
doi:\href{https://doi.org/10.1890/es12-00383.1}{10.1890/es12-00383.1}.

\leavevmode\hypertarget{ref-Asanok2013}{}%
17. Asanok, L., Marod, D., Duengkae, P., Pranmongkol, U., Kurokawa, H.,
Aiba, M., Katabuchi, M. \& Nakashizuka, T. Relationships between
functional traits and the ability of forest tree species to reestablish
in secondary forest and enrichment plantations in the uplands of
northern Thailand. \emph{\textbf{Forest Ecology and Management}} 296,
9--23 (2013),
doi:\href{https://doi.org/10.1016/j.foreco.2013.01.029}{10.1016/j.foreco.2013.01.029}.

\leavevmode\hypertarget{ref-Sasaki2013}{}%
18. Sasaki, T., Katabuchi, M., Kamiyama, C., Shimazaki, M., Nakashizuka,
T. \& Hikosaka, K. Variations in Species Composition of Moorland Plant
Communities Along Environmental Gradients Within a Subalpine Zone in
Northern Japan. \emph{\textbf{Wetlands}} 33, 269--277 (2013),
doi:\href{https://doi.org/10.1007/s13157-013-0380-6}{10.1007/s13157-013-0380-6}.

\leavevmode\hypertarget{ref-Aiba2013}{}%
19. Aiba, M., Katabuchi, M., Takafumi, H., Matsuzaki, S.-i. S., Sasaki,
T. \& Hiura, T. Robustness of trait distribution metrics for community
assembly studies under the uncertainties of assembly processes.
\emph{\textbf{Ecology}} 94, 2873--2885 (2013),
doi:\href{https://doi.org/10.1890/13-0269.1}{10.1890/13-0269.1}.

\leavevmode\hypertarget{ref-Sasaki2012b}{}%
20. Sasaki, T., Katabuchi, M., Kamiyama, C., Shimazaki, M., Nakashizuka,
T. \& Hikosaka, K. Diversity partitioning of moorland plant communities
across hierarchical spatial scales. \emph{\textbf{Biodiversity and
Conservation}} 21, 1577--1588 (2012),
doi:\href{https://doi.org/10.1007/s10531-012-0265-7}{10.1007/s10531-012-0265-7}.

\leavevmode\hypertarget{ref-Sasaki2012}{}%
21. Sasaki, T., Katabuchi, M., Kamiyama, C., Shimazaki, M., Nakashizuka,
T. \& Hikosaka, K. Nestedness and niche-based species loss in moorland
plant communities. \emph{\textbf{Oikos}} 121, 1783--1790 (2012),
doi:\href{https://doi.org/10.1111/j.1600-0706.2012.20152.x}{10.1111/j.1600-0706.2012.20152.x}.

\leavevmode\hypertarget{ref-Katabuchi2012}{}%
22. Katabuchi, M., Kurokawa, H., Davies, S. J., Tan, S. \& Nakashizuka,
T. Soil resource availability shapes community trait structure in a
species-rich dipterocarp forest. \emph{\textbf{Journal of Ecology}} 100,
643--651 (2012),
doi:\href{https://doi.org/10.1111/j.1365-2745.2011.01937.x}{10.1111/j.1365-2745.2011.01937.x}.

\leavevmode\hypertarget{ref-Katabuchi2008b}{}%
23. Katabuchi, M., Isagi, Y. \& Nakashizuka, T. Development of 17
microsatellite markers for Ceratosolen constrictus, the pollinating fig
wasp of Ficus fistulosa. \emph{\textbf{Molecular Ecology Resources}} 8,
840--842 (2008),
doi:\href{https://doi.org/10.1111/j.1755-0998.2007.02084.x}{10.1111/j.1755-0998.2007.02084.x}.

\leavevmode\hypertarget{ref-Katabuchi2008}{}%
24. Katabuchi, M., Harrison, R. D. \& Nakashizuka, T. Documenting the
effect of foundress number in a dioecious fig, Ficus fistulosa, in
Malaysia. \emph{\textbf{Biotropica}} 40, 457--461 (2008),
doi:\href{https://doi.org/10.1111/j.1744-7429.2008.00405.x}{10.1111/j.1744-7429.2008.00405.x}.
\end{cslreferences}

\end{document}
